\useOpTeX
\load[doc]

\tit The `luakeyval` Module
\hfill Version: 0.1, 2025-11-30  \par
\centerline{\it Udi Fogiel, 2025}

\parindent0pt\parskip5pt\parfillskip=20ptplus1fill

The luakeyval LuTeX module is a minimal key/value parser for LuaTeX formats
based on `token.scan_key_cs`. As such, the keword parsing is similar to how \TeX/
parses keywords in special primitives like `\hrule`. 

Unlike \TeX's scanner, `luakeyval` scans key/val pairs inside braces.
This avoids the requirement of appending a `\relax` to the key/val
pairs list to stop \TeX/ from scanning beyond necassery possibly
 creating expansion problems.

\sec Usage
The module provides the `process` function, which accept
a table of keys, and a table of error messages.

Each key in the table should have a table of parameters as a value. The possible
parameters are
\def\param #1:{{\bf #1:}\hskip1em\ignorespaces}
\begitems \style O
* \param scanner: a function that will scan the value of the key from \TeX/ to Lua.
                  The value will usually be a function from \LuaTeX's token library,
                  but you can use your own.
* \param    args: a table of arguments that will be passed to the scanner function. 
* \param default: a value that will return in case the key is not followed by a `=`.
* \param    func: a function that will be executed each time the key appears.
\enditems
Non of the parameters is mandetory, but a key must have at least
one of `default` or `scanner`.


The error messages table can have the following entries
\begitems \style O
* \param         error1: a message that will be displayed if something whent wrong
                         while processing a key/val list. 
* \param         error2: a message that will be displayed after `error1` in case a user
                         press the `H` key for more informatin.
* \param value_required: a message that will be displayed if no value given to a key
                         (when there is no `=` after the key) and the key does not have
                         a default value.
* \param value_forbiden: a message that will be displayed if a key was given a value
                         and the key does not have a `scanner` function.
\enditems

\sec Example
\begtt \hisyntax{lua}
local keyval = require'luakeyval'
local process = keyval.process
local keys = {
    
}
local messages = {

}

\endtt


\sec Important limitations
Since the key/val parser is only a minimal layer
on top of `token.scan_keyword`, key/val pairs should be separated with spaces,
not commas, and you should not define a key which is a prefix of another key.

\sec Implementation
\verbinput \vitt{luakeyval.lua} \commentchars-- \hisyntax{lua} (1-) luakeyval.lua

\bye

